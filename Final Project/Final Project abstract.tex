\documentclass{article}
\renewcommand{\abstractname}{Project Proposal}

\title{Lattice Boltzmann Bhatnaga-Gross-Krook for Simulating Cellular Blood Flow}
\date{24 June 2018}
\author{Group J\\ Rajesh Maheswaran\\ Rohit Abraham Paul\\ Fukushi Sato}

\begin{document}
\maketitle

\begin{abstract}

\noindent
Lattice Boltzmann method (LBM) is an alternative class of CFD methods which simulate fluid behavior. Unlike the finite-element method (FEM) used in the lab, LBM does not solve the Navier-Stokes equations (NSE). Instead, LBM is based on a discrete lattice and solves the Boltzmann equations. Particle interactions are modeled through collisions and the governing forces are made to be consistent with the NSE. LBM is a very successful method which can be used to model turbulence, multi-phase flows, and microscopic phenomena. 

\noindent
Our project proposal is to use the lattice Boltzmann method to simulate cellular blood flow. Our aim will be to model local artery vessel wall irregularities, also called aneurysms. The results of this project can be used to help detect and diagnose potentially fatal aneurysms.
	
\noindent
Our project will be separated into three major components. The first phase will be developing and implementing a lattice Boltzmann Bhatnagar-Gross-Krook (LBBGK) scheme. The second phase will be to implement domain decoupling for complex geometries. The final phase will be benchmarking and analyzing simulations results. If time permits, we will parallelize our implementation and discuss performance and efficiency.
\end{abstract}


\end{document}